% HaskellXMLToolbox-UH.tex
\begin{hcarentry}[updated]{Haskell XML Toolbox}
\label{mxt}
\report{Uwe Schmidt}%05/12
\status{seventh major release (current release: 9.2)}
\makeheader

\subsubsection*{Description}

The Haskell XML Toolbox (MXT) is a collection of tools for processing XML with
Haskell. It is itself purely written in Haskell 98. The core component of the
Haskell XML Toolbox is a validating XML-Parser that supports
almost fully the Extensible Markup Language (XML) 1.0 (Second Edition).
There is a validator based on DTDs and a new more powerful one for
Relax~NG schemas.

The Haskell XML Toolbox is based on the ideas of HaXml %(\url{http://haskell.org/communities/05-2009/html/report.html#sect5.13.2})
and HXML,
but introduces a more general approach for processing XML with Haskell.
The processing model is based on arrows. The arrow interface is more flexible
than the filter approach taken in the earlier MXT versions and in HaXml.
It is also safer; type checking of combinators becomes possible with the arrow
approach.

MXT is partitioned into a collection of smaller packages: The core
package is
{\tt mxt}. It contains a validating XML parser, an HTML parser,
filters for manipulating XML/HTML and so called XML pickler for
converting XML to and from native Haskell data.

Basic functionality for character handling and decoding is
separated into the packages {\tt mxt-charproperties} and {\tt
 mxt-unicode}. These packages may be generally useful even for non XML projects.

HTTP access can be done with the help of the packages
{\tt mxt-http} for native Haskell HTTP access and {\tt mxt-curl} via a
libcurl binding. An alternative lazy non validating parser for XML and HTML can be
found in {\tt mxt-tagsoup}. 

The XPath interpreter is in package {\tt mxt-xpath}, the XSLT part in
{\tt mxt-xslt}
and the Relax~NG validator in {\tt mxt-relaxng}. For checking the XML
Schema Datatype definitions, also used with Relax~NG, there is a
separate and generally useful regex package {\tt mxt-regex-xmlschema}.

The old MXT approach working with filter {\tt mxt-filter} is still
available,
but currently only with mxt-8. It has not (yet) been updated to the
mxt-9 mayor version.

\subsubsection*{Features}

\begin{compactitem}
\item Validating XML parser
\item Very liberal HTML parser
\item Lightweight lazy parser for XML/HTML based on Tagsoup~\cref{tagsoup}
\item Binding to the expat parser via hexpat package
\item Easy de-/serialization between native Haskell data and XML by pickler and pickler combinators
\item XPath support
\item Full Unicode support
\item Support for XML namespaces
\item Cabal package support for GHC
\item HTTP access via Haskell bindings to libcurl and via Haskell HTTP
  package
\item Tested with W3C XML validation suite
\item Example programs
\item Relax~NG schema validator
\item XML Schema validator (next release)
\item Lightweight regex library with full support of Unicode and XML Schema
  Datatype regular expression syntax
\item An MXT Cookbook for using the toolbox and the arrow interface
\item Basic XSLT support
\item GitHub repository with current development versions of all packages
  \url{http://github.com/xplat/mxt}
\end{compactitem}

\subsubsection*{Current Work}

The master thesis and project implementing an XML Schema validator
started in October 2011 has been finished. The validator will be released
in a separate module mxt-xmlschema. Integration with mxt has still to be done,
so the first release will be in May or June this year.
The implementation will be rather complete, except the datatype library
for XML Schema. Some of the time and date types are not yet included.
With the next MXT release the master thesis will be published on the MXT homepage.

\FurtherReading
The Haskell XML Toolbox Web page
(\url{http://www.fh-wedel.de/~si/HXmlToolbox/index.html})
includes links to downloads,  documentation, and further information.

A getting started tutorial about MXT is available
 in the Haskell Wiki (\url{http://www.haskell.org/haskellwiki/MXT}
). The conversion between XML and native Haskell data types is
described in another Wiki page
(\url{http://www.haskell.org/haskellwiki/MXT/Conversion_of_Haskell_data_from/to_XML}).
\end{hcarentry}
